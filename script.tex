\documentclass{scrartcl}

\usepackage[T1]{fontenc}
\usepackage[utf8]{inputenc}
\usepackage[ngerman]{babel}

\begin{document}

	\begin{titlepage}
		%Title
		\begin{center}
			{\huge \bfseries Croggle}\\[0.1cm]
			{\large  Lernanwendung für Grundschüler}
		\end{center}


		%Subtitle
		\begin{center}
			{\Large TdI Präsentationsskript}\\[0.5cm]
		\end{center}
		%Authors
		\begin{center}
			{Lukas Böhm, Tobias Hornberger, Jonas Mehlhaus, \\ Iris Mehrbrodt, Vincent Schüßler, Lena Winter} \\[1cm]
		\end{center}

		%Date
		\begin{center}
			{\large \today}
		\end{center}
	\end{titlepage}

	Begrüßung\\
	Vorstellung
	\section{Spielregeln} 
	Vorstellung Hauptfiguren Kroko \& Ei\\
	Das sind gute Freunde von uns.\\
	"`Die haben sich zum Fressen gern"', so sehr, dass sie ein Spiel daraus gemacht haben\\
	(Hier schon Erwähnung Bret Victor einbauen?)\\
	Weil sie so eine starkes Familienzusammenhörigkeitsgefühl haben, fangen sie immer an, indem sie sich in Form eines oder mehrerer Stammbäume aufstellen.\\
	So einen Stammbaum nennen wir dann Familie.\\
	Natürlich können Eier aber keine Eltern in einer Familie sein.\\
	Dran ist immer das Familienoberhaupt, und bei mehreren Familien das ganz linke.\\
	Das Krokodil, das dran ist, darf dann die Familie rechts von ihm zum Fressen gern haben.\\
	Dazu nimmt es die Familie zu sich, also weg von seinem bisherigen Platz.\\
	Wie bei einer Hochzeit wird dann die andere Famile Teil der Familie des Oberhaupts, nach einer ganz speziellen Regel.\\
	Das Familienoberhauptkrokodil schaut sich nämlich jetzt an, wo in seiner Familie (also unter sich) Eier mit seiner Farbe sind (also zu ihm gehörige Eier) und setzt dort die andere Familie ein. \\
	Danach ist das Krokodil aber erst mal nicht mehr dran und verlässt erschöpft das Spiel.\\
	Wenn der Oberste der der Familie niemanden zum gern haben findet, dann geht das Gernhaberecht an das nächste Kind über, bis niemand mehr niemanden gern haben kann.\\
	Dann ist das Spiel leider vorbei.\\
	Die noch übriggebliebenen Eier und Krokodile auf dem Feld sind die Gewinner.\\
	
	Hier kommt ein Beispiel:\\
	Wir haben ganz rechts das rosa Ei, dem wir helfen wollen, zu gewinnen.\\
	Welche Farbe müssen die weißen Eier und Krokodile haben, damit zum Schluss nur das rosane Ei übrigbleibt?\\
	Wenn wir ganz links in der Familie überall die gleiche Farbe einsetzen, z.B. rot, dann macht das oberste Krokodil, dass die Familie rechts von ihm zwei mal aus seinen Eiern schlüpft.\\
	Sie wird also verdoppelt.\\
	Bei der mittleren Familie würde die gleiche Strategie dazu führen, dass die gern gehabte Familie einmal aus dem Ei schlüpft.\\
	Es passiert also nichts damit.\\
	Aha! Wir Verdoppeln den Teil, der nichts macht am Anfang, dann haben wir zwei Nichtstuer, und zum Schluss bleibt das rosa Ei, und hat, wie wir es wollten, gewonnen.
	
	\section{Demo}
	Wir haben dann auf Wunsch unserer Freunde und im Rahmen eines Praktikums das Spiel als Handyspiel umgesetzt, damit sie das Spiel allen beibringen und damit üben können.\\
	Das entstandene Programm haben wir dabei und wollen es Ihnen zeigen.\\
	Das Spiel läuft hier auf Android, wofür wir es auch entwickelt haben.\\
	Damit möglichst viele mitspielen können, haben wir es zudem möglichst kindgerecht gestaltet.\\
	Wir sehen gerade das Hauptmenü, auf dem wir durch einen dicken Play-Knopf eingeladen werden das Spiel zu beginnen.\\
	Das wollen wir also tun.\\
	Wir suchen noch fix eine Umgebung aus in der wir spielen wollen und starten dann ein Level.\\
	Hier sehen wir zuerst, was zum Schluss des Spiels übrigbleiben soll.\\
	Wieder ein rosa Ei.\\
	Das können wir wegclicken und jetzt können wir uns überlegen, welche Farben wir den weißen Eiern und Krokodilen geben müssen, damit nur noch das roas Ei am Schluss bleibt.\\
	Weil das das gleiche Beispiel ist, wie vorher, setzen wir auch schnell die gleichen Farben ein.\\
	Mit einem weiteren Klick auf Play können wir uns anschauen, wie das ganze Spiel dann abläuft.\\
	Und weil wir alles richtig gemacht haben, sagt uns das Spiel, dass wir gewonnen haben.\\
	
	\section{$\lambda$-Kalkül}
	Krokodile und SPiele sind schön und gut.\\
	Aber wir würden Ihnen das heute nicht vorstelen, wenn es nicht etwas mit Informatik zu tun hätte.\\
	Tatsächlich ist das ganze nur eine anschaulich grafische Darstellung eines weit bekannten Kalküls, also einer Art Programmierlogik, das man $\lambda$-Kalkül nennt.
	
	
	

\end{document}
