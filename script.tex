\documentclass{scrartcl}

\usepackage[T1]{fontenc}
\usepackage[utf8]{inputenc}
\usepackage[ngerman]{babel}

\begin{document}

	\begin{titlepage}
		%Title
		\begin{center}
			{\huge \bfseries Croggle}\\[0.1cm]
			{\large  Lernanwendung für Grundschüler}
		\end{center}


		%Subtitle
		\begin{center}
			{\Large TdI Präsentationsskript}\\[0.5cm]
		\end{center}
		%Authors
		\begin{center}
			{Lukas Böhm, Tobias Hornberger, Jonas Mehlhaus, \\ Iris Mehrbrodt, Vincent Schüßler, Lena Winter} \\[1cm]
		\end{center}

		%Date
		\begin{center}
			{\large \today}
		\end{center}
	\end{titlepage}

	Begrüßung\\
	Vorstellung
	\section{Spielregeln}
	Hier sehen wir unsere Hauptfiguren Kroko \& Ei.
	"`Die haben sich zum Fressen gern"'.
	Weil sie so eine starkes Familienzusammenhörigkeitsgefühl haben, fangen sie immer an, indem sie sich in Form eines oder mehrerer Stammbäume aufstellen.
	So einen Stammbaum nennen wir dann Familie.
	Natürlich können Eier aber keine Eltern in einer Familie sein.
	Sieht ein Krokodil vor sich ein anderes Krokodil, so bekommt es großen Hunger. So schmackhaft wie das andere Krokodil aussieht, kann es der Versuchung nicht lange wiederstehen und frisst mit einem Haps das Krokodil samt dessen ganzer Familie.
	Leider bekommt es dem fressenden Krokodil nicht besonders, eine ganze Familie auf einmal zu verschlingen, was zu seinem verfrühten Ableben führt.
	Zum Glück hat unser Krokodil aber bereits für Nachwuchs gesorgt: Die Eier unseres Krokodils, zu erkennen an der gleichen Farbe, beginnen im Moment seines Ablebens zu schlüpfen!
	Überraschenderweise sehen die neugeborenen Kinder unseres verschiedenen Krokodils genauso aus, wie die verschlungene Familie.
	Dieses Spektakel wiederholt sich, bis kein Krokodil mehr etwas Fressbares vor der Schnauze hat.

	% (\x . \y . x y) (\z . z) x: grün, y: rot, z: gelb
	Am besten sieht man das an einem einfachen Beispiel.
	Hier sind zwei Familien zu sehen: Das grüne Krokodil beschützt das rote Krokodil und die beiden Eier, das gelbe Krokodil das gelbe Ei.
	Nun sieht das grüne Krokodil das gelbe und es passiert das unvermeidliche: Die gelbe Familie wird vom grünen Krokodil verschlungen.
	Daraufhin stirbt das grüne Krokodil, und aus dem grünen Ei schlüpft eine neue gelbe Familie!
	Jetzt aber wird das gelbe Krokodil hungrig und frisst das benachbarte rote Ei.
	Es stirbt ebenfalls und aus dem verbleibenden gelben Ei schlüpft ein rotes Ei.
	Damit kann kein Krokodil mehr etwas fressen und es passiert nichts mehr.

	% (\x . x x) (\x . x x) x: grün
	Das muss allerdings nicht immer so sein.
	Hier zu sehen sind zwei identische grüne Familien.
	Was passiert nun?
	Das linke Krokodil frisst die rechte Familie.
	Es stirbt, und aus beiden Eiern schlüpft jeweils die gefressene Familie.
	Das Ergebnis haben wir aber schon mal gesehen, es ist nämlich genau die Ausgangsstellung.
	Wir wissen, was jetzt als nächstes passiert, nämlich noch einmal genau das gleiche wie gerade eben.
	Es gibt also kein Ende und das Fressen setzt sich immer weiter fort.

	% (\x . x x)(\y . y) z x,y: weiß, z: rosa
	Man könnte sich aber auch andere Fragen stellen.
	Zum Beispiel ein Rätsel, das ungefähr so aussieht:
	Welche Farbe müssen die weißen Eier und Krokodile haben, damit zum Schluss nur das rosane Ei übrigbleibt?
	Bevor wir aber zur Lösung dieses Rätsels kommen, möchten wir Ihnen vorstellen, was wir daraus gemacht haben.

	%Wenn wir ganz links in der Familie überall die gleiche Farbe einsetzen, z.B. rot, dann macht das oberste Krokodil, dass die Familie rechts von ihm zwei mal aus seinen Eiern schlüpft.
	%Sie wird also verdoppelt.
	%Bei der mittleren Familie würde die gleiche Strategie dazu führen, dass die gern gehabte Familie einmal aus dem Ei schlüpft.
	%Es passiert also nichts damit.
	%Aha! Wir Verdoppeln den Teil, der nichts macht am Anfang, dann haben wir zwei Nichtstuer, und zum Schluss bleibt das rosa Ei übrig.

	\section{Demo}
	Wir haben die Krokodile und ihre Eier im Rahmen eines Praktikums als Lernspiel für Smartphones und Tablets mit dem Android Betriebssystem umgesetzt, damit alle damit üben können.
	Besonders wichtig war uns dabei eine möglichst kindgerechte Gestaltung.
	Das entstandene Programm haben wir dabei und wollen es Ihnen nun zeigen.

	Wir sehen gerade das Hauptmenü, auf dem wir durch einen dicken Play-Knopf eingeladen werden das Spiel zu beginnen.
	Das wollen wir also tun.
	Wir suchen nun eine Umgebung aus in der wir spielen wollen und starten dann ein Level.
	Hier sehen wir zuerst, was zum Schluss des Spiels übrigbleiben soll: ein rosa Ei.
	Wenn wir diese Zielstellung schließen, sehen wir das bekannte Rätsel von vorhin.
	Jetzt können wir uns überlegen, welche Farben wir den weißen Eiern und Krokodilen geben müssen, damit nur noch das rosa Ei am Schluss bleibt.
	Färben wir die linke Familie in einer Farbe ein zB blau, so wird die Familie, die rechts davon steht, verdoppelt.
	Die mittlere Familie wird nun von uns gelb eingefärbt.
	Sehen wir uns das Resultat an.
	Da aus den gelben Eiern immer das schlüpft, was die gelben Alligatoren gefressen haben, bleibt am Ende nur das rosa Ei übrig, also haben wir gewonnen. 
	
	
	

	\section{$\lambda$-Kalkül}
	Krokodile und Spiele sind wunderbar.
	Aber wir würden Ihnen das heute nicht vorstellen, wenn es nicht etwas mit Informatik zu tun hätte.
	Tatsächlich ist das ganze nur eine von Bret Victor erdachte, anschauliche grafische Darstellung eines weit bekannten Kalküls, also einer Art Programmierlogik, den man $\lambda$-Kalkül nennt.
	Grob gesagt bedeutet das, dass man beim Spielen des Spiels an die Denkweisen der Programmierung herangeführt wird und diese erlernt.
	Tatsächlich wird das Spielprinzip am KIT bereits für Lehrveranstaltungen des fünften Semesters verwendet.

	In seiner Grundform wirkt der $\lambda$-Kalkül deutlich abschreckender, als es Krokodile tun.
	Die vorhin gezeigte Fressregel heißt im $\lambda$-Kalkül $\beta$-Reduktion.
	Krokodile entsprechen $\lambda$-Abstraktionen, Eier entsprechen Variablen und die Farben entsprechen den Variablennamen.

	Zur Verdeutlichung haben wir hier die beiden Darstellungen miteinander verglichen.
	Auf der linken Seite steht der gleiche Ausdruck als Krokodilfamilie wie rechts als $\lambda$-Ausdruck.
	Nun beginnt auf beiden Seiten die Auswertung.
	Das grüne Krokodil frisst wieder das gelbe Ei, was zum wegfallen der Abstraktion bzw. zum Tod des Krokodils führt.
	Gleichzeitig wird das Ei, bzw. die Variable, durch das andere ersetzt.
	Schließlich ist das Auswertungsergebnis auf beiden Seiten äquivalent.

	\section{Softwaretechnik}
	Zur Umsetzung unseres Projekts konnten wir auf einige hilfreiche Werkzeuge zurückgreifen.
	Als Programmiersprache kam Java zum Einsatz, sowie Eclipse mit Androidunterstützung als integrierte Entwicklungsumgebung und git zur Versionskontrolle.
	In der Entwurfsphase haben wir zur Erstellung von UML-Diagrammen UMLet und ArgoUML verwendet.
	Als übergeordnetes Entwurfsmuster haben wir uns sehr stark an Model-View-Controller orientiert.
	Um die Testüberdeckung unseres Projektes zu überprüfen haben wir mit Emma und dem zugehörigen Eclipse-Plugin EclEmma gearbeitet.

	Insgesamt umfasst das Projekt etwa 13000 Zeilen Code, der sich auf etwa 170 Klassen aufteilt.
	Hinzu kommen etwa 250 Testfälle, die aus 4600 Zeilen Java Code bestehen, sowie 600 Zeilen Python für Integrationstests.
	Damit konnten wir eine Testfallabdeckung von etwa 96\% im Model und 78\% insgesamt erreichen.

	\section{Projekterfahrungen}

	Während der Entwicklung von Croggle haben wir wertvolle Erfahrungen darüber gesammelt wie man gemeinsam größere Projekte stemmen kann. Bis zum PSE ist das Studium eher theoretisch und von Einzelleistungen geprägt, deswegen bietet das PSE den Studenten den besonderen Anreiz den kompletten Entwicklungszyklus eines Softwareprojektes mit anderen Studenten und unter professioneller Betreuung durch die Institute zu durchlaufen. \\
	Gerade die kreative Freiheit, die mit unserer Aufgabenstellung verbunden war, hat uns sehr gefallen: wir konnten unsere eigene Ideen umsetzen und so dem Projekt eine persönliche Note geben. \\


	%Die vielen Herausforderungen und Probleme die es im Projektverlauf zu überwinden gilt haben uns als Gruppe näher zusammen gebracht.
	%Für uns alle war das PSE eine wertvolle und neue Erfahrung. Natürlich verschlingt ein so großes Projekt sehr viel Freizeit, aber dafür macht es auch umso mehr Spaß.



	%Das PSE ist bei den meisten Studenten, auch bei uns sehr beliebt.
	%Es ist das erste große Softwareprojekt das man im Studium macht, zusätzlich ist es in Gruppenarbeit und unter professioneller Betreuung durch die Institute zu beältigen was zum einen den Spaß als auch den Lerneffekt vergrößert.\\
	%Gerade die kreative Freiheit die mit unserer Aufgabenstellung verbunden war hat uns sehr gefallen, wir konnten unsere eigene Ideen umsetzen uns so lernen wie man sowas korrekt implementiert. \\
	Uns zumindestens hat es sehr großen Spaß gemacht auch wenn wir sehr viel unserer Freizeit dafür geopfert haben. \\
	Wir konnten sehr viele wertvolle Erfahrungen sammeln, die sich sicherlich auch in Zukunft noch als nützlich erweisen werden. 	
	An dieser Stelle möchten wir auch nochmals unseren Betreuern danken, die uns bei unseren Ideen unterstützt haben und uns gegebenenfalls auch auf kleine Fehler unsereres Entwurfes hingewiesen haben. \\
     

\end{document}
