\documentclass{scrartcl}

\usepackage[T1]{fontenc}
\usepackage[utf8]{inputenc}
\usepackage[ngerman]{babel}

\begin{document}

	\begin{titlepage}
		%Title
		\begin{center}
			{\huge \bfseries Croggle}\\[0.1cm]
			{\large  Lernanwendung für Grundschüler}
		\end{center}


		%Subtitle
		\begin{center}
			{\Large TdI Präsentationsskript}\\[0.5cm]
		\end{center}
		%Authors
		\begin{center}
			{Lukas Böhm, Tobias Hornberger, Jonas Mehlhaus, \\ Iris Mehrbrodt, Vincent Schüßler, Lena Winter} \\[1cm]
		\end{center}

		%Date
		\begin{center}
			{\large \today}
		\end{center}
	\end{titlepage}

	Begrüßung\\
	Vorstellung
	\section{Spielregeln} 
	Vorstellung Hauptfiguren Kroko \& Ei\\
	Das sind gute Freunde von uns.\\
	"`Die haben sich zum Fressen gern"', so sehr, dass sie ein Spiel daraus gemacht haben\\
	(Hier schon Erwähnung Bret Victor einbauen?)\\
	Weil sie so eine starkes Familienzusammenhörigkeitsgefühl haben, fangen sie immer an, indem sie sich in Form eines oder mehrerer Stammbäume aufstellen.\\
	So einen Stammbaum nennen wir dann Familie.\\
	Natürlich können Eier aber keine Eltern in einer Familie sein.\\
	Dran ist immer das Familienoberhaupt, und bei mehreren Familien das ganz linke.\\
	Das Krokodil, das dran ist, darf dann die Familie rechts von ihm zum Fressen gern haben.\\
	Und dabei wird dann (wie in einem echten Stammbaum) die gern gehabte Familie Teil der ersten Familie.\\
	Das Krokodil schaut sich nämlich jetzt an, wo in seiner Familie Eier mit seiner Farbe sind (also seine Eier) und setzt dort die andere Familie ein. \\
	Danach ist das Krokodil aber erst mal nicht mehr dran und verlässt erschöpft das Spiel.\\
	Wenn jemand niemanden zum gern haben findet, dann geht das Gernhaberecht das nächste Kind über.
	
	\section{Demo}
	Wir haben dann auf Wunsch unserer Freunde und im Rahmen eines Praktikums das Spiel als Handyspiel umgesetzt.\\
	Das haben wir dabei und wollen es Ihnen zeigen.
	

\end{document}
