\documentclass{scrartcl}

\usepackage[T1]{fontenc}
\usepackage[utf8]{inputenc}
\usepackage[ngerman]{babel}

\begin{document}

	\begin{titlepage}
		%Title
		\begin{center}
			{\huge \bfseries Croggle}\\[0.1cm]
			{\large  Lernanwendung für Grundschüler}
		\end{center}


		%Subtitle
		\begin{center}
			{\Large TdI Präsentationsskript}\\[0.5cm]
		\end{center}
		%Authors
		\begin{center}
			{Lukas Böhm, Tobias Hornberger, Jonas Mehlhaus, \\ Iris Mehrbrodt, Vincent Schüßler, Lena Winter} \\[1cm]
		\end{center}

		%Date
		\begin{center}
			{\large \today \\
				Sprecher 1: Lena \\
				Sprecher 2:	Tobi \\
				Sprecher 3: Lukas \\
				Sprecher 4: Vincent
			}
		\end{center}
	\end{titlepage}

	\section{Begrüßung und Vorstellung: S1}
	Hallo und herzlich willkommen zur PSE Vorstellung der Gruppe Croggle.
	Wir sind: Lukas, Tobias, Jonas, Iris, Vincent und ich bin Lena.
	Wir möchten ihnen nun das Projekt vorstellen, an dem wir das gesamte letzte Semester gemeinsam gearbeiten haben.

	\section{Spielregeln: S2} 
	Hier sehen wir unsere Hauptfiguren Kroko \& Ei.
	"`Die haben sich zum Fressen gern"'.
	Weil Krokodile und Eier so ein starkes Zusammengehörigkeitsgefühl haben, stellen sie sich immer in Form eines oder mehrerer Stammbäume auf.
	So einen Stammbaum nennen wir dann eine Familie.
	Hier sehen wir 2 solche Familien.	
	Sieht ein Krokodil vor sich ein anderes Krokodil, so bekommt es großen Hunger, wie hier das Grüne. So schmackhaft wie das andere Krokodil aussieht, kann unser grüner Freund der Versuchung nicht lange wiederstehen und frisst mit einem Haps das Krokodil samt dessen ganzer Familie auf.
	Leider bekommt das dem grünen Krokodil nicht besonders, was zu seinem verfrühten Ableben führt.
	Zum Glück hat unser Krokodil aber bereits für Nachwuchs gesorgt: Das grüne Ei unseres Krokodils beginnt im Moment seines Ablebens zu schlüpfen!
	Überaschenderweise sehen die neugeborenen Kinder unseres toten Krokodils genauso aus, wie die verschlungene Familie.
	Dieses Spektakel wiederholt sich, bis kein Krokodil mehr etwas Fressbares vor der Schnauze hat.

	% (\x . x x)(\y . y) z x,y: weiß, z: rosa
	Dazu kann man sich auch Rätsel überlegen, z.B.:
	Welche Farbe müssen die weißen Eier und Krokodile haben, damit zum Schluss nur das rosane Ei übrigbleibt?
	Bevor wir aber zur Lösung dieses Rätsels kommen, möchten wir Ihnen vorstellen, was wir daraus gemacht haben.

	%Wenn wir ganz links in der Familie überall die gleiche Farbe einsetzen, z.B. rot, dann macht das oberste Krokodil, dass die Familie rechts von ihm zwei mal aus seinen Eiern schlüpft.
	%Sie wird also verdoppelt.
	%Bei der mittleren Familie würde die gleiche Strategie dazu führen, dass die gern gehabte Familie einmal aus dem Ei schlüpft.
	%Es passiert also nichts damit.
	%Aha! Wir Verdoppeln den Teil, der nichts macht am Anfang, dann haben wir zwei Nichtstuer, und zum Schluss bleibt das rosa Ei übrig.

	\section{Demo: S3}
	Wir haben die Krokodile und ihre Eier im Rahmen eines Praktikums als Lernspiel für Smartphones und Tablets mit dem Android Betriebssystem umgesetzt, damit alle damit üben können.
	Besonders wichtig war uns dabei eine möglichst kindgerechte Gestaltung.
	Das entstandene Programm haben wir dabei und wollen es Ihnen nun zeigen.

	Wir sehen gerade das Hauptmenü, auf dem wir durch einen dicken Play-Knopf eingeladen werden das Spiel zu beginnen.
	Das wollen wir also tun.
	Wir suchen nun eine Umgebung aus in der wir spielen wollen und starten dann ein Level.
	Hier sehen wir zuerst, was zum Schluss des Spiels übrigbleiben soll: ein rosa Ei.
	Wenn wir diese Zielstellung schließen, sehen wir das bekannte Rätsel von vorhin.
	Jetzt können wir uns überlegen, welche Farben wir den weißen Eiern und Krokodilen geben müssen, damit nur noch das rosa Ei am Schluss bleibt.
	Färben wir die linke Familie in einer Farbe ein zB blau, so wird die Familie, die rechts davon steht, verdoppelt.
	Die mittlere Familie wird nun von uns gelb eingefärbt.
	Sehen wir uns das Resultat an.
	Da aus den gelben Eiern immer das schlüpft, was die gelben Alligatoren gefressen haben, bleibt am Ende nur das rosa Ei übrig, also haben wir gewonnen. 
	
	
	

	\section{$\lambda$-Kalkül: S4}
	Krokodile und Spiele sind wunderbar.
	Aber wir würden Ihnen das heute nicht vorstellen, wenn es nicht etwas mit Informatik zu tun hätte.
	Tatsächlich ist das ganze nur eine von Bret Victor erdachte, anschauliche grafische Darstellung eines weit bekannten Kalküls, also einer Art Programmierlogik, den man $\lambda$-Kalkül nennt.
	Grob gesagt bedeutet das, dass man beim Spielen des Spiels an die Denkweisen der Programmierung herangeführt wird und diese erlernt.
	Tatsächlich wird das Spielprinzip am KIT bereits für Lehrveranstaltungen des fünften Semesters verwendet.

	In seiner Grundform wirkt der $\lambda$-Kalkül deutlich abschreckender, als es Krokodile tun.
	Die vorhin gezeigte Fressregel heißt im $\lambda$-Kalkül $\beta$-Reduktion.
	Krokodile entsprechen $\lambda$-Abstraktionen, Eier entsprechen Variablen und die Farben entsprechen den Variablennamen.

	Zur Verdeutlichung haben wir hier die beiden Darstellungen miteinander verglichen.
	Auf der linken Seite steht der gleiche Ausdruck als Krokodilfamilie wie rechts als $\lambda$-Ausdruck.
	Nun beginnt auf beiden Seiten die Auswertung.
	Das grüne Krokodil frisst wieder das gelbe Ei, was zum wegfallen der Abstraktion bzw. zum Tod des Krokodils führt.
	Gleichzeitig wird das Ei, bzw. die Variable, durch das andere ersetzt.
	Schließlich ist das Auswertungsergebnis auf beiden Seiten äquivalent.

	\section{Softwaretechnik: S1}
	Zur Umsetzung unseres Projekts konnten wir auf einige hilfreiche Werkzeuge zurückgreifen:
	Dazu gehören Java, Eclipse mit Androidunterstützung, git, UMLet, Emma und natürlich noch vieles mehr.

	Insgesamt umfasst das Projekt etwa 13000 Zeilen Code, der sich auf etwa 170 Klassen aufteilt.
	Hinzu kommen etwa 250 Testfälle, die aus 4600 Zeilen Java Code bestehen, sowie 600 Zeilen Python für Integrationstests.
	Damit konnten wir eine Testfallabdeckung von etwa 96\% im Model und 78\% insgesamt erreichen.

	\section{Projekterfahrungen: S1}

	Während der Entwicklung von Croggle haben wir wertvolle Erfahrungen darüber gesammelt wie man gemeinsam größere Projekte stemmen kann. 
	Gerade die kreative Freiheit, die mit unserer Aufgabenstellung verbunden war, hat uns sehr gefallen: wir konnten unsere eigene Ideen umsetzen und so dem Projekt eine persönliche Note geben. \\
	Auch wenn wir viel unserer Freizeit dafür geopfert haben, hat es uns doch allen sehr großen Spaß gemacht.
	Verabschiedung: S3
     

\end{document}
